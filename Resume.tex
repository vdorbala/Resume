\documentclass{resume}
\usepackage{hyperref}
\usepackage[left=0.75in,top=0.6in,right=0.75in,bottom=0.6in]{geometry}
\newcommand{\tab}[1]{\hspace{.2667\textwidth}\rlap{#1}}
\newcommand{\itab}[1]{\hspace{0em}\rlap{#1}}
\name{Vishnu Sashank Dorbala}
\address{Website: \href{https://vdorbala.github.io}{https://vdorbala.github.io}}
\address{(+91)7743875800 \\ \href{mailto:vdorbala@gmail.com}{vdorbala@gmail.com}}
\usepackage{ifthen}
% % from https://tex.stackexchange.com/questions/209627/how-to-disable-list-in-the-rsubsection
% \newenvironment{rSubsection}[4]{% 4 input arguments - company name, year(s) employed, job title and location
%  {\bf #1} \hfill {#2} % Bold company name and date on the right
%  \ifthenelse{\equal{#3}{}}{}{ % If the third argument is not specified, don't print the job title and location line
%   \\
%   {\em #3} \hfill {\em #4} % Italic job title and location
%   }\smallskip
%   \begin{list}{$\cdot$}{\leftmargin=0em} % \cdot used for bullets, no indentation
%   \itemsep -0.5em \vspace{-0.5em} % Compress items in list together for aesthetics
%   }{
%   \end{list}
%   \vspace{0.5em} % Some space after the list of bullet points
% }

\begin{document}
\noindent%

\begin{rSection}{Research Interests}

Robotics, Computer Vision, Machine Learning, Autonomous Navigation, Human-Robot Interaction, Social \& Assistive Robotics, Reinforcement Learning, Visual Servoing.
\end{rSection}

\begin{rSection}{Education}

{\bf Symbiosis International University, Pune} \hfill {\em July 2013 - July 2017} 
\\ Bachelor of Technology, \hfill {CGPA: 3.3/4}
\\ Electronics and Telecommunications Engineering \\
\\{\bf Ascent Classes, Visakhapatnam} \hfill {\em June 2011 - June 2013} 
\\ Class XII, SSC Board \hfill { Aggregate: 91\% }

\end{rSection}

\begin{rSection}{Technical Skills}

\begin{tabular}{ @{} >{\bfseries}l @{\hspace{6ex}} l }
Programming Languages \ & Python, MATLAB, C/C++ \\
Libraries \& Tools & Pytorch, Tensorflow, OpenCV, Slurm, LaTeX, MaZda \\
Softwares \ & Robot Operating System (RoS), Gazebo, RviZ 
\end{tabular}

\end{rSection}

\begin{rSection}{Research Experiences \& Projects}

\begin{rSubsection}{Center for Visual Information Technology, IIIT, Hyderabad}{Sept 2017 - April 2019}{Research Fellow under Prof. C.V. Jawahar}{}

{\textbullet \hspace{0.1em} \textbf{Intelligent Wheelchair Platform/Semi Autonomous Robot Assistant (SARA)}}

\vspace*{-0.2em}
\begin{list}{$\cdot$}{\leftmargin=1.2em}
\itemsep-0.1em 
\item Developed an Intelligent Wheelchair Platform as a low cost retrofittable solution to help wheelchair users move autonomously.
Remodelled a standard power wheelchair for the purpose, by retrofitting it with a Kinect v2 for perception and a Sabertooth Motor driver for motion control.
\item Used RoS along with RTABMAP and RviZ for deploying individual modules for mapping, re-localization, path planning, and navigation onto the platform. 
\item Guided undergraduate honour students on developing modules for facial recognition, person following, object recognition and tracking, and an android application to utilize the platform.
\item Platform aptly renamed SARA, short for Semi-Autonomous Robot Assistant, when the scope of the project expanded to investigate Human-Robot Interaction (HRI). Currently investigating a social robotics task on SARA, where the robot would gain Social Intelligence by learning to take appropriate actions by analyzing images of how people interact in front of it.
\end{list}

{\textbullet \hspace{0.1em} \textbf{Deep Learning for Visual Servoing}}

\vspace*{-0.2em}
\begin{list}{$\cdot$}{\leftmargin=1.2em}
\itemsep0em 
\item Conducted research on incorporating Deep Neural Networks for performing Image Based Visual Servoing (IBVS) in a corridor following task. Was jointly guided by \href{https://scholar.google.com/citations?user=U9dH-DoAAAAJ&hl=en}{Prof. C.V. Jawahar} and \href{https://scholar.google.com/citations?user=81UGxdYAAAAJ&hl=en}{Prof. A.H. Abdul Hafez}.
\item Designed experiments for using CNNs to directly predict a velocity vector from a captured image to servo a wheelchair robot across a corridor.
\item Used SARA as a testbed for performing practical corridor following experiments across different hallways in IIIT-H.
\item A paper on this work was accepted at the \textbf{International Conference on Intelligent Robots and Systems (IROS)}, 2019.
\hfill
(\href{https://vdorbala.github.io/pdf/IROS_paper.pdf}{Paper}, \href{https://vdorbala.github.io/pdf/IROS_video.mp4}{Video})
\item Part of this work was accepted into a Turkish national conference, SIU 2019.
\hfill
(\href{https://vdorbala.github.io/pdf/SIU2019.pdf}{Paper})
\end{list}
\hfill
\end{rSubsection}
\begin{rSubsection}{Symbiosis International University, Pune}{June 2013 - June 2017}{Undergraduate Student}{}
\item

{\textbullet \hspace{0.1em} \textbf{Classifying Brain Tumor Mutations (BTech. Thesis)}}

\vspace*{-0.2em}
\begin{list}{$\cdot$}{\leftmargin=1em}
\item The project was aimed at providing an efficient means for classifying brain tumor mutations using Image Processing, as an alternative to an invasive surgical procedure involved otherwise. Was actively mentored by \href{https://scholar.google.com/citations?hl=en&user=UdKti2kAAAAJ&view_op=list_works&sortby=pubdate}{Dr. Madhura Ingalhalikar}.
\item Performed image texture analysis on brain tumor MRI scans using a tool called MaZda and extracted over 100 textural features. Adopted a Naive Bayes model for classification, after performing dimensionality reduction via Principal Component Analysis on the extracted features.
\end{list}

{\textbullet \hspace{0.1em} \textbf{ABU Robocon 2016}}

\vspace*{-0.2em}
\begin{list}{$\cdot$}{\leftmargin=1em}
\item Led our university robotics team called "Symbotics" for 9 months in my junior year on constructing two differential drive inter-dependant (one driving the other using wind energy) robots for the national rounds of Robocon 2016, a prestigious international robotics competition.
\item Conceptualized and constructed several different configurations for the robots, and designed circuitry for robot control, energy storage and sensor integration. 
\item Conducted research by trial and error on optimizing the energy stored in a supercapacitor circuit on one of the robots that was harnessing energy captured from a wind source. Reached an adequately optimal state by introducing a circuit to appropriately switch between parallel and series configurations of the supercapacitor circuit by taking cues from an inertial sensor.
\item As team leader, responsibilities included distribution of tasks, making critical choices on design, and compiling progress reports. Being on a limited budget, also managed the team finances and kept accounts on expenditure.
\end{list}

{\textbullet \hspace{0.1em} \textbf{Datagram Transport Layer Security (DTLS) for IoT devices}}

\vspace*{-0.2em}
\begin{list}{$\cdot$}{\leftmargin=1em}
\item Interned for 6 months at a startup called Graphene Semiconductors Inc., and worked on developing a library in embedded C for implementing DTLS on IoT devices.
\item Wrote abstracted scripts for implementing cookie exchanges, handling endianness, Diffie-Hellman key exchange protocols and server side multi-threading.
\end{list}
\end{rSubsection}
\end{rSection}

\begin{rSection}{Extra-Curricular}
\begin{itemize}
    \item Conducted a workshop at IIIT-H's Computer Vision Summer School on `3D Vision and Structure for Motion (SfM)'. Created assignments to evaluate the participants' performance as part of this.
    \item Designed and organized a hackathon on `Autonomous Navigation' for an AI/ML course for industry professionals held at IIIT-H. Used Gazebo to create a simulated obstacle track for a Turtlebot to navigate. Wrote template code for the participants to plug in their CNN models trained to predict direction from a signboard image to make the Turtlebot autonomously navigate through the simulated track. Later, recreated this in a real world environment.
    \item Stood first in the regional round of a Haptic Robotics competition conducted by IIT Bombay. Constructed a robot that used hand movements detected via potentiometers to operate a miniature crane system.
\end{itemize}
\end{rSection}

\end{document}